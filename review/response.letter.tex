\documentclass[journal,onecolumn,12pt]{IEEEtran} 

\usepackage{amsmath,amssymb,bm}
\usepackage{amsthm, amsfonts}	
\usepackage{bm,bbm}
\usepackage[normalem]{ulem}
\usepackage{color}
\usepackage{fancybox}
\usepackage{url,booktabs}
\usepackage[round]{natbib}

\usepackage{xr}



\title{Reply to Reviewer's Comments on\\
``A Control Strategy for a Tethered Follower Robot for Pulmonary Rehabilitation''}
\author{}

\begin{document}

\maketitle
\pagenumbering{roman}
\setcounter{page}{1}

We are grateful to all the reviewer for  the time taken to pointing out relevant issues in our manuscript.

In the following, we discuss how we dealt with each raised issue. 

\vskip+1ex
\noindent \dotfill

\section*{\fbox{Associate Editor Transcript:}}

Comments to the Author:
Authors addressed an important problem, providing a simple yet effective solution. There is no doubt that the device presented in the paper is of great interest for both patients and clinicians. To publish the paper on a robotics journal, though, the technical content has to be deepened so to increase the interest of researchers.
As better detailed by the the two reviewers, the Authors are encouraged to improve the manuscript along two main directions:

\begin{itemize}
\item Clarifying the novelties wrt to state of the art (robot design and robot control), to highlight the original and novel contributions;
\item  Improving the modelling, which should take into account realistic working conditions (e.g. weight and inertial properties of the tank).
\end{itemize}


\section*{\fbox{Reviewer \#1 Transcript:}}

I found the work interesting and tackling an important problem, which can have applications outside support in this specific task. The need to carry oxygen tanks around patients with respiratory problems is high in hospitals and rehabilitation facilities. This solution frees nurses or other medical staff that would be occupied with such tasks and even patients that could carry the tank themselves with such solution can have an easier life in these places.

\vspace{2em}

Overall the article is very well written, I've found some typos to which I'll refer in a momment. The idea is transmitted clearly and the authors did an excellent job motivating the problem and covering the relevant literature. The prototype defined solves the problem in an elegant and simple way and I find the documentation on how the robot was designed sufficient. However, Section II.D, the control strategy section, could have explored more the strategies as some variables and expressions are not entirely clear regarding their aim; despite that the reader can understand how each strategy works and their differences.

\vspace{2em}

Regarding the evaluation and discussion I don't have much to add, I believe you conducted two solid validations of the system that showed the system performs adequately and the conclusions from those studies seem in line with both the results and the clinical assessment performed. My only question is why were the configurations used in the real world study not part of the set of configurations used in the simulation? I understand why you reduced the sets of configurations used in the real world study, but I would expect that those sets would come from the simulations ran not entirely new sets of configurations. I think this part should have been better explained.

\vspace{2em}

Finally, the typos I found:
      in page 2, left side, line 22, at the begin of section II - "Hence, design methodologies that allow rapid prototyping can bring quickly feedback from...", I think it sould be "quick" instead of "quickly"
      in page 3, left side, line 46, at the begin of section II.C - "... and to save battery,
an algorithm to activate and deactivate the motors is developed.", it should be "motors was developed" and not "is"
      in page 7, left side, line 19 - "Both control strategies were expounded, along with the main superficial differences between them.", did you mean "expounded" or "explained"

\vspace{2em}

\section*{\fbox{Reviewer \#2 Transcript:}}

The authors propose a differential tethered robot to follow a COPD patient during rehabilitation. The robot carries the heavy oxygen tank that the patient cannot carry on their own. The tethered robot uses two threads attached to motorized reels to control the robots position w.r.t the patient as the patient performs various rehabilitation exercises. The authors propose two control strategies and conduct both simulated and real-world experiments.

\vspace{2em}

Overall, the reviewer found the application to be interesting; it seems like a great opportunity for a robotic solution. The experiments conducted appear to be reasonably selected. It is unclear, however, what the significance of the work is, particularly in the context of previous work [1] which also investigates a tethered robot solution for COPD rehabilitation. The discernible differences between the proposed work and that within [1] appear to be that the proposed robot has two tethers instead of one, and that the solution can be used in medical facilities instead of home therapy environments (the latter of which would probably have even more complicated navigation environments). There is no comparison between this existing work and the proposed work which, given they are essentially tackling the same application, is problematic: what are the advantages/disadvantages of one design over another? Is there something specifically wrong with the design in [1] that prompted the authors to develop their own approach?

\vspace{2em}

[1]  Endo, Gen, et al. "Mobile follower robot as an assistive device for home oxygen therapy–evaluation of tether control algorithms." ROBOMECH Journal 2.1 (2015): 1-9.

\vspace{2em}


There were also a number of technical elements the reviewer had a hard time understanding:
\begin{itemize}
\item How are "pulses" defined (equation (1)), and how is this value used in an equation?
\item How does a distance multiplied by a constant yield a velocity? (equation (2)); It would seem there is some element of time missing here.  
\item How does a velocity, plus a distance multiplied by a constant yield a power value? (equations (3) and (4)); additionally, using "v" to denote both velocity and power seems like a poor notation choice
\end{itemize}

\vspace{2em}

In addition to these comments, it would seem that a robot being designed to carry a heavy oxygen tank would be designed with said tank in mind. The current calibration of constants and experimental results are not done with the weight of a heavy oxygen tank, and thus it is hard to determine if the design and control techniques would actually succeed in their intended application. Is there a reason that a weight of some kind was not used for the experiments? Can non-weighted experiments be justified somehow?

\vspace{2em}

Given the issues of significance w.r.t. existing work and technical questions raised, the reviewer recommends major revisions.  

\vspace{2em}

Detailed minor comments:

\begin{itemize}
\item - The abstract and introduction (specifically) suffer from English/grammar issues. (e.g. "eskeletomuscular" to musculoskeletal); the reviewer recommends the paper be peer-edited by a native speaker if it has not been already.
\item - The footer says the paper has been "revised August 26, 2020", which is of course a date that has not happened yet.
\item - The term $Dt{off}$ is often spelled $Dt{o}ff$ and should be fixed (same with $Dm{off}$).
\item - The caption for Table 1 should be with the actual table, not in a separate column.
\item - In Figure 5 it is not visible that a marker is placed on the side of the robot, nor on the hand of the patient (also the patient is not wearing a glove which is mentioned in the paper).
\end{itemize}


Page 6, first paragraph: "a high cr was important so the robot can turn quickly to point to the leader, but high values of cr should be avoided"; this seems contradictory.

\section*{\fbox{Associate Editor Responses:}}

Comments to the Author:
Authors addressed an important problem, providing a simple yet effective solution. There is no doubt that the device presented in the paper is of great interest for both patients and clinicians. To publish the paper on a robotics journal, though, the technical content has to be deepened so to increase the interest of researchers.
As better detailed by the the two reviewers, the Authors are encouraged to improve the manuscript along two main directions:

\begin{itemize}
\item Clarifying the novelties wrt to state of the art (robot design and robot control), to highlight the original and novel contributions;
\item  Improving the modelling, which should take into account realistic working conditions (e.g. weight and inertial properties of the tank).
\end{itemize}


\begin{quotation}
{\color{blue}
We appreciate for the Associate Editor's comments.  We concentrated our efforts modifying the manuscript in the two important lines that the Editor recommended.  

We addressed the first issue by adding a new Section "Comparison with alternative methods" where we compared the tethered controller with one and two threads and compared our proposal with other alternative proposals from the literature.  We also included an evaluation of technical aspects of the design.

Regarding the second issue, we performed an additional experiment on the simulation environment where we included a 5 kg mass, in a 45 degrees orientation, according to the weight and size of the oxygen tank regularly used at ALPI.  We verified in the simulation that we obtained results very similar to the those obtained without the tank and that the controlling algorithm worked within the safe boundaries that we defined and that the following behavior was satisfied.
}
\end{quotation}


\section*{\fbox{Reviewer \#1 Responses:}}

%\subsection*{\ovalbox{Reviewer 1 General Comments}}

I found the work interesting and tackling an important problem, which can have applications outside support in this specific task. The need to carry oxygen tanks around patients with respiratory problems is high in hospitals and rehabilitation facilities. This solution frees nurses or other medical staff that would be occupied with such tasks and even patients that could carry the tank themselves with such solution can have an easier life in these places.

Overall the article is very well written, I've found some typos to which I'll refer in a momment. The idea is transmitted clearly and the authors did an excellent job motivating the problem and covering the relevant literature. The prototype defined solves the problem in an elegant and simple way and I find the documentation on how the robot was designed sufficient. However, Section II.D, the control strategy section, could have explored more the strategies as some variables and expressions are not entirely clear regarding their aim; despite that the reader can understand how each strategy works and their differences.

\begin{quotation}
{\color{blue}
We thanks very much, the Reviewer, for the positive appraisal.  We revised the notation on Section II.D.  We also added a section to establish the parallels between a single and double thread configuration, comparing the solution to other alternatives found in the literature.  We clarified all the variables and their physical meaning and included an extra Figure to aid in the provided explanation.
}
\end{quotation}


Regarding the evaluation and discussion I don't have much to add, I believe you conducted two solid validations of the system that showed the system performs adequately and the conclusions from those studies seem in line with both the results and the clinical assessment performed. My only question is why were the configurations used in the real world study not part of the set of configurations used in the simulation? I understand why you reduced the sets of configurations used in the real world study, but I would expect that those sets would come from the simulations ran not entirely new sets of configurations. I think this part should have been better explained.

\begin{quotation}
{\color{blue}
During the simulation we explored different sets of values for the controller parameters.  Once we arrived to parameter values where we obtain a range of satisfactory results in terms of the metrics that we defined, we moved forward to the real world testing of the prototype.  We found that we had to fine-tune the parameters to adapt them to the real world scenario.  However, the relative relation between parameters was consistent on both the simulation and in the real world, and that helped us to avoid searching extensively the parameter space for the real world scenario.  We added in the manuscript a clarification of this point in Section VI.
}
\end{quotation}


Finally, the typos I found:
      in page 2, left side, line 22, at the begin of section II - "Hence, design methodologies that allow rapid prototyping can bring quickly feedback from...", I think it sould be "quick" instead of "quickly"
      
 \begin{quotation}
{\color{blue}
Fixed
}
\end{quotation}     
      
      in page 3, left side, line 46, at the begin of section II.C - "... and to save battery, an algorithm to activate and deactivate the motors is developed.", it should be "motors was developed" and not "is"

\begin{quotation}
{\color{blue}
Fixed
}
\end{quotation}

      in page 7, left side, line 19 - "Both control strategies were expounded, along with the main superficial differences between them.", did you mean "expounded" or "explained"
      
      
      \begin{quotation}
{\color{blue}
We changed to "explained", which we believe is clearer.  Thanks for the suggestion.
}
\end{quotation}

\section*{\fbox{Reviewer \#2 Responses:}}
%\subsection*{\ovalbox{Reviewer 2 General Comments}}

The authors propose a differential tethered robot to follow a COPD patient during rehabilitation. The robot carries the heavy oxygen tank that the patient cannot carry on their own. The tethered robot uses two threads attached to motorized reels to control the robots position w.r.t the patient as the patient performs various rehabilitation exercises. The authors propose two control strategies and conduct both simulated and real-world experiments.

\vspace{2em}

Overall, the reviewer found the application to be interesting; it seems like a great opportunity for a robotic solution. The experiments conducted appear to be reasonably selected. It is unclear, however, what the significance of the work is, particularly in the context of previous work [1] which also investigates a tethered robot solution for COPD rehabilitation. The discernible differences between the proposed work and that within [1] appear to be that the proposed robot has two tethers instead of one, and that the solution can be used in medical facilities instead of home therapy environments (the latter of which would probably have even more complicated navigation environments). There is no comparison between this existing work and the proposed work which, given they are essentially tackling the same application, is problematic: what are the advantages/disadvantages of one design over another? Is there something specifically wrong with the design in [1] that prompted the authors to develop their own approach?

\vspace{2em}

[1]  Endo, Gen, et al. "Mobile follower robot as an assistive device for home oxygen therapy–evaluation of tether control algorithms." ROBOMECH Journal 2.1 (2015): 1-9.

\begin{quotation}
{\color{blue}
We added a new specific section to address this issue.  Now in Section III "Comparison with alternative methods", we show that the proposed algorithm based on two threads is equivalent to the one presented in the work [1].  We added a new Figure 5 to aid in the explanation.  Additionally, we included an additional "Design Analysis" section comparing both alternatives (two threads, and one thread and an angle).  Moreover, we modified the notation to allow an improved comparison with previous proposals.
}
\end{quotation}

There were also a number of technical elements the reviewer had a hard time understanding:
- How are "pulses" defined (equation (1)), and how is this value used in an equation?

\begin{quotation}
{\color{blue}
We refer to pulses as a discrete number obtained directly from the rotary encoder KY-040.  The encoder generates two digital pulses, one that indicates the presence of movement and the other is the direction of movement.  This digital encoder is connected to the microcontroller and it generates an interruption each time the encoder produces a rotational movement.  By reading the additional pulse, the direction can be determined and the value for pulses represents the relative circular distance traveled by the reel wheel.   We added the definition of pulses in the manuscript on Section II.A.
}
\end{quotation}


- How does a distance multiplied by a constant yield a velocity? (equation (2)); It would seem there is some element of time missing here.  

\begin{quotation}
{\color{blue}
We appreciate a lot this important issue that the Reviewer is pointing out here.  We modified the notation trying to make it more consistent across the paper and more consistent with previous works as well.  We clarify the units in the proportional gain $c_v$ and $c_{\alpha}$ to make equations physically meaningful.
}
\end{quotation}


- How does a velocity, plus a distance multiplied by a constant yield a power value? (equations (3) and (4)); additionally, using "v" to denote both velocity and power seems like a poor notation choice

\begin{quotation}
{\color{blue}
We added proper units for the constants and modify the equations clarifying that left-hand values reference speeds for each wheel.
}
\end{quotation}

In addition to these comments, it would seem that a robot being designed to carry a heavy oxygen tank would be designed with said tank in mind. The current calibration of constants and experimental results are not done with the weight of a heavy oxygen tank, and thus it is hard to determine if the design and control techniques would actually succeed in their intended application. Is there a reason that a weight of some kind was not used for the experiments? Can non-weighted experiments be justified somehow?

\begin{quotation}
{\color{blue}
We performed an additional simulation and added the obtained results in Section VII.B, at the end of the manuscript.  We modified our simulated prototype and added a $5$ kg oxygen tank at an inclination of 45 degrees, and confirmed that the oxygen supply is not affected by this inclination, and that is under safe usage guidelines for the tank.  The proposed algorithm \textit{Follow the thread} worked in a very similar way as shown in Figure 7, and only constant coefficients had to be readjusted for this purpose.  Results show that the behavior of the robot is similar.  This is a very important feasibility test that allows and encourage us to modify the hardware prototype to include the tank holder, and to move forward to this project's next step.
}
\end{quotation}

Given the issues of significance w.r.t. existing work and technical questions raised, the reviewer recommends major revisions.  

Detailed minor comments:

- The abstract and introduction (specifically) suffer from English/grammar issues. (e.g. "eskeletomuscular" to musculoskeletal); the reviewer recommends the paper be peer-edited by a native speaker if it has not been already.

\begin{quotation}
{\color{blue}
We replaced the term with musculoskeletal and verified the English grammar on the Abstract and Introduction section. 
}
\end{quotation}

- The footer says the paper has been "revised August 26, 2020", which is of course a date that has not happened yet.

\begin{quotation}
{\color{blue}
We are very surprised the the fast review process!   We truly thank all the Reviewers for their valuable time.  Dates have now been fixed.
}
\end{quotation}

- The term $Dt{off}$ is often spelled $Dt{o}ff$ and should be fixed (same with $Dm{off}$).

\begin{quotation}
{\color{blue}
We reviewed all the constants definitions and naming conventions to be consistent along the manuscript.
}
\end{quotation}

- The caption for Table 1 should be with the actual table, not in a separate column.

\begin{quotation}
{\color{blue}
This is an issue with Latex reformatting.  We fixed it by modifying the layout of the manuscript.
}
\end{quotation}

- In Figure 5 it is not visible that a marker is placed on the side of the robot, nor on the hand of the patient (also the patient is not wearing a glove which is mentioned in the paper).

\begin{quotation}
{\color{blue}
There are two markers on top of the robot, just above each reel.  They are quite small and the picture lighting does not help.  We modified the Figure caption clarifying all this information.
}
\end{quotation}

Page 6, first paragraph: "a high $c_r$ was important so the robot can turn quickly to point to the leader, but high values of $c_r$ should be avoided"; this seems contradictory.

\begin{quotation}
{\color{blue}
We change the phrase to emphasize that the value of $c_r$ was related with the dynamical behavior of the robot.  We appreciate the Reviewers comment.
}
\end{quotation}


\vskip+1ex
\noindent \dotfill
\vskip+1ex
\bibliographystyle{mdpi}
\bibliography{article}

\end{document}