
%% bare_jrnl.tex
%% V1.4b
%% 2015/08/26
%% by Michael Shell
%% see http://www.michaelshell.org/
%% for current contact information.
%%
%% This is a skeleton file demonstrating the use of IEEEtran.cls
%% (requires IEEEtran.cls version 1.8b or later) with an IEEE
%% journal paper.
%%
%% Support sites:
%% http://www.michaelshell.org/tex/ieeetran/
%% http://www.ctan.org/pkg/ieeetran
%% and
%% http://www.ieee.org/

%%*************************************************************************
%% Legal Notice:
%% This code is offered as-is without any warranty either expressed or
%% implied; without even the implied warranty of MERCHANTABILITY or
%% FITNESS FOR A PARTICULAR PURPOSE! 
%% User assumes all risk.
%% In no event shall the IEEE or any contributor to this code be liable for
%% any damages or losses, including, but not limited to, incidental,
%% consequential, or any other damages, resulting from the use or misuse
%% of any information contained here.
%%
%% All comments are the opinions of their respective authors and are not
%% necessarily endorsed by the IEEE.
%%
%% This work is distributed under the LaTeX Project Public License (LPPL)
%% ( http://www.latex-project.org/ ) version 1.3, and may be freely used,
%% distributed and modified. A copy of the LPPL, version 1.3, is included
%% in the base LaTeX documentation of all distributions of LaTeX released
%% 2003/12/01 or later.
%% Retain all contribution notices and credits.
%% ** Modified files should be clearly indicated as such, including  **
%% ** renaming them and changing author support contact information. **
%%*************************************************************************


% *** Authors should verify (and, if needed, correct) their LaTeX system  ***
% *** with the testflow diagnostic prior to trusting their LaTeX platform ***
% *** with production work. The IEEE's font choices and paper sizes can   ***
% *** trigger bugs that do not appear when using other class files.       ***                          ***
% The testflow support page is at:
% http://www.michaelshell.org/tex/testflow/



\documentclass[journal]{IEEEtran}
%
% If IEEEtran.cls has not been installed into the LaTeX system files,
% manually specify the path to it like:
% \documentclass[journal]{../sty/IEEEtran}





% Some very useful LaTeX packages include:
% (uncomment the ones you want to load)


% *** MISC UTILITY PACKAGES ***
%
%\usepackage{ifpdf}
% Heiko Oberdiek's ifpdf.sty is very useful if you need conditional
% compilation based on whether the output is pdf or dvi.
% usage:
% \ifpdf
%   % pdf code
% \else
%   % dvi code
% \fi
% The latest version of ifpdf.sty can be obtained from:
% http://www.ctan.org/pkg/ifpdf
% Also, note that IEEEtran.cls V1.7 and later provides a builtin
% \ifCLASSINFOpdf conditional that works the same way.
% When switching from latex to pdflatex and vice-versa, the compiler may
% have to be run twice to clear warning/error messages.






% *** CITATION PACKAGES ***
%
%\usepackage{cite}
% cite.sty was written by Donald Arseneau
% V1.6 and later of IEEEtran pre-defines the format of the cite.sty package
% \cite{} output to follow that of the IEEE. Loading the cite package will
% result in citation numbers being automatically sorted and properly
% "compressed/ranged". e.g., [1], [9], [2], [7], [5], [6] without using
% cite.sty will become [1], [2], [5]--[7], [9] using cite.sty. cite.sty's
% \cite will automatically add leading space, if needed. Use cite.sty's
% noadjust option (cite.sty V3.8 and later) if you want to turn this off
% such as if a citation ever needs to be enclosed in parenthesis.
% cite.sty is already installed on most LaTeX systems. Be sure and use
% version 5.0 (2009-03-20) and later if using hyperref.sty.
% The latest version can be obtained at:
% http://www.ctan.org/pkg/cite
% The documentation is contained in the cite.sty file itself.






% *** GRAPHICS RELATED PACKAGES ***
%
\ifCLASSINFOpdf
  % \usepackage[pdftex]{graphicx}
  % declare the path(s) where your graphic files are
  % \graphicspath{{../pdf/}{../jpeg/}}
  % and their extensions so you won't have to specify these with
  % every instance of \includegraphics
  % \DeclareGraphicsExtensions{.pdf,.jpeg,.png}
\else
  % or other class option (dvipsone, dvipdf, if not using dvips). graphicx
  % will default to the driver specified in the system graphics.cfg if no
  % driver is specified.
  % \usepackage[dvips]{graphicx}
  % declare the path(s) where your graphic files are
  % \graphicspath{{../eps/}}
  % and their extensions so you won't have to specify these with
  % every instance of \includegraphics
  % \DeclareGraphicsExtensions{.eps}
\fi
% graphicx was written by David Carlisle and Sebastian Rahtz. It is
% required if you want graphics, photos, etc. graphicx.sty is already
% installed on most LaTeX systems. The latest version and documentation
% can be obtained at: 
% http://www.ctan.org/pkg/graphicx
% Another good source of documentation is "Using Imported Graphics in
% LaTeX2e" by Keith Reckdahl which can be found at:
% http://www.ctan.org/pkg/epslatex
%
% latex, and pdflatex in dvi mode, support graphics in encapsulated
% postscript (.eps) format. pdflatex in pdf mode supports graphics
% in .pdf, .jpeg, .png and .mps (metapost) formats. Users should ensure
% that all non-photo figures use a vector format (.eps, .pdf, .mps) and
% not a bitmapped formats (.jpeg, .png). The IEEE frowns on bitmapped formats
% which can result in "jaggedy"/blurry rendering of lines and letters as
% well as large increases in file sizes.
%
% You can find documentation about the pdfTeX application at:
% http://www.tug.org/applications/pdftex





% *** MATH PACKAGES ***
%
\usepackage{amsmath}
\usepackage{amsfonts}
\usepackage{booktabs}
% A popular package from the American Mathematical Society that provides
% many useful and powerful commands for dealing with mathematics.
%
% Note that the amsmath package sets \interdisplaylinepenalty to 10000
% thus preventing page breaks from occurring within multiline equations. Use:
\interdisplaylinepenalty=2500
% after loading amsmath to restore such page breaks as IEEEtran.cls normally
% does. amsmath.sty is already installed on most LaTeX systems. The latest
% version and documentation can be obtained at:
% http://www.ctan.org/pkg/amsmath





% *** SPECIALIZED LIST PACKAGES ***
%
%\usepackage{algorithmic}
% algorithmic.sty was written by Peter Williams and Rogerio Brito.
% This package provides an algorithmic environment fo describing algorithms.
% You can use the algorithmic environment in-text or within a figure
% environment to provide for a floating algorithm. Do NOT use the algorithm
% floating environment provided by algorithm.sty (by the same authors) or
% algorithm2e.sty (by Christophe Fiorio) as the IEEE does not use dedicated
% algorithm float types and packages that provide these will not provide
% correct IEEE style captions. The latest version and documentation of
% algorithmic.sty can be obtained at:
% http://www.ctan.org/pkg/algorithms
% Also of interest may be the (relatively newer and more customizable)
% algorithmicx.sty package by Szasz Janos:
% http://www.ctan.org/pkg/algorithmicx




% *** ALIGNMENT PACKAGES ***
%
%\usepackage{array}
% Frank Mittelbach's and David Carlisle's array.sty patches and improves
% the standard LaTeX2e array and tabular environments to provide better
% appearance and additional user controls. As the default LaTeX2e table
% generation code is lacking to the point of almost being broken with
% respect to the quality of the end results, all users are strongly
% advised to use an enhanced (at the very least that provided by array.sty)
% set of table tools. array.sty is already installed on most systems. The
% latest version and documentation can be obtained at:
% http://www.ctan.org/pkg/array


% IEEEtran contains the IEEEeqnarray family of commands that can be used to
% generate multiline equations as well as matrices, tables, etc., of high
% quality.




% *** SUBFIGURE PACKAGES ***
%\ifCLASSOPTIONcompsoc
%  \usepackage[caption=false,font=normalsize,labelfont=sf,textfont=sf]{subfig}
%\else
%  \usepackage[caption=false,font=footnotesize]{subfig}
%\fi
% subfig.sty, written by Steven Douglas Cochran, is the modern replacement
% for subfigure.sty, the latter of which is no longer maintained and is
% incompatible with some LaTeX packages including fixltx2e. However,
% subfig.sty requires and automatically loads Axel Sommerfeldt's caption.sty
% which will override IEEEtran.cls' handling of captions and this will result
% in non-IEEE style figure/table captions. To prevent this problem, be sure
% and invoke subfig.sty's "caption=false" package option (available since
% subfig.sty version 1.3, 2005/06/28) as this is will preserve IEEEtran.cls
% handling of captions.
% Note that the Computer Society format requires a larger sans serif font
% than the serif footnote size font used in traditional IEEE formatting
% and thus the need to invoke different subfig.sty package options depending
% on whether compsoc mode has been enabled.
%
% The latest version and documentation of subfig.sty can be obtained at:
% http://www.ctan.org/pkg/subfig




% *** FLOAT PACKAGES ***
%
%\usepackage{fixltx2e}
% fixltx2e, the successor to the earlier fix2col.sty, was written by
% Frank Mittelbach and David Carlisle. This package corrects a few problems
% in the LaTeX2e kernel, the most notable of which is that in current
% LaTeX2e releases, the ordering of single and double column floats is not
% guaranteed to be preserved. Thus, an unpatched LaTeX2e can allow a
% single column figure to be placed prior to an earlier double column
% figure.
% Be aware that LaTeX2e kernels dated 2015 and later have fixltx2e.sty's
% corrections already built into the system in which case a warning will
% be issued if an attempt is made to load fixltx2e.sty as it is no longer
% needed.
% The latest version and documentation can be found at:
% http://www.ctan.org/pkg/fixltx2e


%\usepackage{stfloats}
% stfloats.sty was written by Sigitas Tolusis. This package gives LaTeX2e
% the ability to do double column floats at the bottom of the page as well
% as the top. (e.g., "\begin{figure*}[!b]" is not normally possible in
% LaTeX2e). It also provides a command:
%\fnbelowfloat
% to enable the placement of footnotes below bottom floats (the standard
% LaTeX2e kernel puts them above bottom floats). This is an invasive package
% which rewrites many portions of the LaTeX2e float routines. It may not work
% with other packages that modify the LaTeX2e float routines. The latest
% version and documentation can be obtained at:
% http://www.ctan.org/pkg/stfloats
% Do not use the stfloats baselinefloat ability as the IEEE does not allow
% \baselineskip to stretch. Authors submitting work to the IEEE should note
% that the IEEE rarely uses double column equations and that authors should try
% to avoid such use. Do not be tempted to use the cuted.sty or midfloat.sty
% packages (also by Sigitas Tolusis) as the IEEE does not format its papers in
% such ways.
% Do not attempt to use stfloats with fixltx2e as they are incompatible.
% Instead, use Morten Hogholm'a dblfloatfix which combines the features
% of both fixltx2e and stfloats:
%
% \usepackage{dblfloatfix}
% The latest version can be found at:
% http://www.ctan.org/pkg/dblfloatfix




%\ifCLASSOPTIONcaptionsoff
%  \usepackage[nomarkers]{endfloat}
% \let\MYoriglatexcaption\caption
% \renewcommand{\caption}[2][\relax]{\MYoriglatexcaption[#2]{#2}}
%\fi
% endfloat.sty was written by James Darrell McCauley, Jeff Goldberg and 
% Axel Sommerfeldt. This package may be useful when used in conjunction with 
% IEEEtran.cls'  captionsoff option. Some IEEE journals/societies require that
% submissions have lists of figures/tables at the end of the paper and that
% figures/tables without any captions are placed on a page by themselves at
% the end of the document. If needed, the draftcls IEEEtran class option or
% \CLASSINPUTbaselinestretch interface can be used to increase the line
% spacing as well. Be sure and use the nomarkers option of endfloat to
% prevent endfloat from "marking" where the figures would have been placed
% in the text. The two hack lines of code above are a slight modification of
% that suggested by in the endfloat docs (section 8.4.1) to ensure that
% the full captions always appear in the list of figures/tables - even if
% the user used the short optional argument of \caption[]{}.
% IEEE papers do not typically make use of \caption[]'s optional argument,
% so this should not be an issue. A similar trick can be used to disable
% captions of packages such as subfig.sty that lack options to turn off
% the subcaptions:
% For subfig.sty:
% \let\MYorigsubfloat\subfloat
% \renewcommand{\subfloat}[2][\relax]{\MYorigsubfloat[]{#2}}
% However, the above trick will not work if both optional arguments of
% the \subfloat command are used. Furthermore, there needs to be a
% description of each subfigure *somewhere* and endfloat does not add
% subfigure captions to its list of figures. Thus, the best approach is to
% avoid the use of subfigure captions (many IEEE journals avoid them anyway)
% and instead reference/explain all the subfigures within the main caption.
% The latest version of endfloat.sty and its documentation can obtained at:
% http://www.ctan.org/pkg/endfloat
%
% The IEEEtran \ifCLASSOPTIONcaptionsoff conditional can also be used
% later in the document, say, to conditionally put the References on a 
% page by themselves.




% *** PDF, URL AND HYPERLINK PACKAGES ***
%
%\usepackage{url}
% url.sty was written by Donald Arseneau. It provides better support for
% handling and breaking URLs. url.sty is already installed on most LaTeX
% systems. The latest version and documentation can be obtained at:
% http://www.ctan.org/pkg/url
% Basically, \url{my_url_here}.




% *** Do not adjust lengths that control margins, column widths, etc. ***
% *** Do not use packages that alter fonts (such as pslatex).         ***
% There should be no need to do such things with IEEEtran.cls V1.6 and later.
% (Unless specifically asked to do so by the journal or conference you plan
% to submit to, of course. )


% correct bad hyphenation here
\hyphenation{op-tical net-works semi-conduc-tor}


\begin{document}
%
% paper title
% Titles are generally capitalized except for words such as a, an, and, as,
% at, but, by, for, in, nor, of, on, or, the, to and up, which are usually
% not capitalized unless they are the first or last word of the title.
% Linebreaks \\ can be used within to get better formatting as desired.
% Do not put math or special symbols in the title.
\title{A Signal Averaging Procedure based on Gambini's Algorithm}
%
%
% author names and IEEE memberships
% note positions of commas and nonbreaking spaces ( ~ ) LaTeX will not break
% a structure at a ~ so this keeps an author's name from being broken across
% two lines.
% use \thanks{} to gain access to the first footnote area
% a separate \thanks must be used for each paragraph as LaTeX2e's \thanks
% was not built to handle multiple paragraphs
%

\author{Rodrigo~Ramele,~\IEEEmembership{Member,~IEEE,}
        Juliana~Gambini,
        and~Juan~Miguel~Santos% <-this % stops a space
\thanks{R. Ramele, J.Gambini and J.M.Santos are with the Department
of  Computer Engineering, Instituto Tecn{o'}gico de Buenos Aires (ITBA), Ciudad de Buenos Aires,  Argentina e-mail: rramele@itba.edu.ar}% <-this % stops a space
\thanks{Manuscript received April 19, 2005; revised August 26, 2015.}}

% note the % following the last \IEEEmembership and also \thanks - 
% these prevent an unwanted space from occurring between the last author name
% and the end of the author line. i.e., if you had this:
% 
% \author{....lastname \thanks{...} \thanks{...} }
%                     ^------------^------------^----Do not want these spaces!
%
% a space would be appended to the last name and could cause every name on that
% line to be shifted left slightly. This is one of those "LaTeX things". For
% instance, "\textbf{A} \textbf{B}" will typeset as "A B" not "AB". To get
% "AB" then you have to do: "\textbf{A}\textbf{B}"
% \thanks is no different in this regard, so shield the last } of each \thanks
% that ends a line with a % and do not let a space in before the next \thanks.
% Spaces after \IEEEmembership other than the last one are OK (and needed) as
% you are supposed to have spaces between the names. For what it is worth,
% this is a minor point as most people would not even notice if the said evil
% space somehow managed to creep in.



% The paper headers
\markboth{IEEE Transactions on Biomedical Engineering}%
{Ramele \MakeLowercase{\textit{et al.}}: A Signal Averaging Procedure based on Gambini's Algorithm}
% The only time the second header will appear is for the odd numbered pages
% after the title page when using the twoside option.
% 
% *** Note that you probably will NOT want to include the author's ***
% *** name in the headers of peer review papers.                   ***
% You can use \ifCLASSOPTIONpeerreview for conditional compilation here if
% you desire.




% If you want to put a publisher's ID mark on the page you can do it like
% this:
%\IEEEpubid{0000--0000/00\$00.00~\copyright~2015 IEEE}
% Remember, if you use this you must call \IEEEpubidadjcol in the second
% column for its text to clear the IEEEpubid mark.



% use for special paper notices
%\IEEEspecialpapernotice{(Invited Paper)}




% make the title area
\maketitle

% As a general rule, do not put math, special symbols or citations
% in the abstract or keywords.
\begin{abstract}
EEG
BCI
Signal Averaging
Solutions
What lies behind.

\end{abstract}

% Note that keywords are not normally used for peerreview papers.
\begin{IEEEkeywords}
IEEE, IEEEtran, journal, \LaTeX, paper, template.
\end{IEEEkeywords}






% For peer review papers, you can put extra information on the cover
% page as needed:
% \ifCLASSOPTIONpeerreview
% \begin{center} \bfseries EDICS Category: 3-BBND \end{center}
% \fi
%
% For peerreview papers, this IEEEtran command inserts a page break and
% creates the second title. It will be ignored for other modes.
\IEEEpeerreviewmaketitle



\section{Introduction}
% The very first letter is a 2 line initial drop letter followed
% by the rest of the first word in caps.
% 
% form to use if the first word consists of a single letter:
% \IEEEPARstart{A}{demo} file is ....
% 
% form to use if you need the single drop letter followed by
% normal text (unknown if ever used by the IEEE):
% \IEEEPARstart{A}{}demo file is ....
% 
% Some journals put the first two words in caps:
% \IEEEPARstart{T}{his demo} file is ....
% 
% Here we have the typical use of a "T" for an initial drop letter
% and "HIS" in caps to complete the first word.

%The EEG is traditionally analyzed in terms of temporal waveforms at certain channels, looking at power of rhythms in the spontaneous EEG, at amplitude and latency of the peaks and troughs in event- related potentials (ERPs), or at particular grapho-elements in patho- logical or sleep stages.

\IEEEPARstart{A}{lthough} recent advances in neuroimagining techniques, particularly radio-nuclear and radiological scanning methods \cite{Schomer2010}, have diminished the prospects of the traditional Electroencephalography (EEG), the advent and development of digitized devices has impelled for a revamping of this hundred years old technology.  Their versatility, ease of use, temporal resolution, ease of development and production, and its proliferation as consumer devices, are pushing EEG to become the de-facto non invasive portable or ambulatory method to access and harness brain information~\cite{DeVos2014}.

A key contribution to this expansion has been the field of Brain Computer Interfaces (BCI)~\cite{WolpawJonathanR2012} which is the pursuit of the development of a new channel of communication particularly aimed to persons affected by neurodegenerative diseases.

One noteworthy aspect of this novel communication channel is the ability to transmit information from the Central Nervous System (CNS) to a computer device and from there use that information to control a wheelchair~\cite{Carlson2013}, as input to a speller application~\cite{Guger2009a}, in a Virtual Reality environment~\cite{Lotte2013} or as aiding tool in a rehabilitation procedure~\cite{Jure2016}.  The holly grail of BCI is to implement a new complete and alternative pathway to restore lost locomotion~\cite{WolpawJonathanR2012}.

EEG signals are remarkably complex and have been characterized as a multichannel non-stationary stochastic process.  Additionally, they have high variability between different subjects and even between different moments for the same subject, requiring adaptive and co-adaptive calibration and learning procedures~\cite{Clerc}.  Hence, this imposes an outstanding challenge that is necessary to overcome in order to extract information from raw EEG signals.

%Moreover, EEG markers~\cite{Clerc} that can be used to  transmit volitional information are limited, and each one of them has a particular combination of appropriate methods to decode them. Inevitably, it is necessary to implement  distinct and specialized algorithmic methods, to filter the signal, enhance its Signal to Noise Ratio (SNR), and try to determine some meaning out of it.  

BCI has gained mainstream public awareness with worldwide challenge competitions like Cybathlon~\cite{Riener2014,cybathlon2} and even been broadcasted during the inauguration ceremony of the 2014 Soccer World Cup.  New developments have overcome the out-of-the-lab high-bar and they are starting to be used in real world environments~\cite{Guger2017,Huggins2016}.  However, they still lack the necessary robustness, and its performance is well behind any other method of human computer interaction, including any kind of detection of residual muscular movement~\cite{Clerc}.

A few works have analyzed the problem of syncrhonizing different signals to average them and to enhance their SNR.

This paper reports a method to, (1) describe a procedure to capture the shape of a waveform of an ERP component, the P300, using histograms of gradient orientations extracted from images of signal plots, and (2) outline the way in which this procedure can be used to implement an P300-Based BCI Speller application. Its validity is verified by offline processing two datasets, one of data from ALS patients and another one from data of healthy subjects. 

This article unfolds as follows: Section~\ref{Feature} is dedicated to explain the Feature Extraction method based on Histogram of Gradient Orientations of the Signal Plot: Section~\ref{Pipeline} shows the preprocessing pipeline,  Section~\ref{Plot}  describes the image generation of the signal plot, Section~\ref{SIFT}  presents the feature extraction procedure while  Section~\ref{Classification}  introduces the Speller Matrix Letter Identification procedure.  In Section~\ref{Protocol}, the experimental protocol is expounded. Section~\ref{Results} shows the results of applying the proposed technique.  In the final Section~\ref{discussion}  we expose our remarks, conclusions and future work.

\subsection{Feature Extraction from Signal Plots} \label{Feature}

In this section, the signal preprocessing, the method for generating images from signal plots, the feature extraction procedure and the Speller Matrix identification are described.  Figure~\ref{fig:classification} shows a scheme of the entire process.

\subsubsection{Preprocessing Pipeline} \label{Pipeline}

The data obtained by the capturing device is digitalized and a multichannel EEG signal is constructed.

%A trial, as defined by the BCI2000 platform~\cite{Schalk2004}, is every attempt to select a letter from the speller. 

%It is composed of signal segments $S_{i}^l$ corresponding to $k_a$ repetitions of flashes of 6 rows and $k_a =10$ repetitions of flashes of 6 columns of the matrix, yielding 120 repetitions. 

The $6$ rows and $6$ columns of the Speller Matrix are intensified providing the visual stimulus.  The number of a row or column is a location. A sequence of twelve randomly permuted locations $l$ conform an intensification sequence. The whole set of twelve intensifications is repeated $k_a$ times.

%The multichannel EEG signal is processed on a channel by channel basis.   

\begin{itemize}
\item \textbf{Signal Enhancement}: This stage consists of the enhancement of the SNR of the P300 pattern above the level of basal EEG. The pipeline starts by applying a notch filter to the raw digital signal, a $4$th degree $10$ Hz lowpass Butterworth filter and finally a decimation with a Finite Impulse Response (FIR) filter of order $30$ from the original sampling frequency down to $16$ Hz \cite{Krusienski2006}.
\item \textbf{Artifact Removal}: For every complete sequence of $12$ intensifications of $6$ rows and $6$ columns, a basic artifact elimination procedure is implemented by removing the entire sequence when any signal deviates above/bellow $ \pm 70 \mu V $.
\item \textbf{Segmentation}: For each of the $12$ intensifications of one intensification sequence,  a segment $S_{i}^l$  of a window of $t_{max} $ seconds of the multichannel signal is extracted, starting from the stimulus onset, corresponding to each row/column intensification $l$ and to the intensification sequence $i$. As intensifications are permuted in a random order, the segments are rearranged corresponding to row flickering, labeled 1-6, whereas those corresponding to column flickering are labeled 7-12.  Two of these segments should contain the P300 ERP signature time-locked to the flashing stimulus, one for the row, and one for the column.
\item \textbf{Signal Averaging}: \label{Average}  The P300 ERP is deeply buried under basal EEG so the standard approach to identify it is by point-to-point averaging the time-locked stacked signal segments.  Hence the values which are not related to, and not time-locked to the onset of the stimulus are canceled out~\cite{Liang2008}.  

This last step determines the operation of any P300 Speller.  In order to obtain an improved signal in terms of its SNR,  repetitions of the sequence of row/column intensification are necessary.  And, at the same time, as long as more repetitions are needed, the ability to transfer information faster is diminished, so there is a trade-off that must be acutely determined.

The procedure to obtain the point-to-point averaged signal goes as follows:

\begin{enumerate}
\item \label{paso1}Highlight randomly the rows and columns from the matrix.  There is one row and one column that should match the letter selected by the subject.
\item  \label{paso2} Repeat step~\ref{paso1} $k_a$ times, obtaining the $1 \leq l \leq 12$ segments $S_1^l(n,c),\dots,S_{k_a}^l(n,c)$, of the EEG signal where the variables $1 \leq n \leq n_{max}$ and $1 \leq c \leq C$  correspond to sample points and channel, respectively. The parameter $C$ is the number of available EEG channels whereas $n_{max}=F_s \  t_{max}$ is the segment length and $F_s$ is the sampling frequency.  The parameter $k_a$ is the number of repetitions of intensifications and it is an input parameter of the algorithm.
\item \label{paso3} Compute the Ensemble Average by
\begin{equation}
x^l(n,c)= \frac{1}{k_a}\sum_{i=1}^{k_a}S_i^l(n,c) 
\label{averaging}
\end{equation}  
for $1 \leq n \leq n_{max}$ and for the channels $1 \leq c \leq C$.  This provide an averaged signal $x^l(n,c)$ for the twelve locations $ 1 \leq l \leq 12$.
\end{enumerate}
\end{itemize}

\subsubsection{Signal Plotting} \label{Plot}

Averaged signal segments are standardized and scaled for $1 \leq n \leq n_{max}$ and $1 \leq c \leq C$ by 

\begin{equation}
\tilde{x}^l(n,c) = \left \lfloor{ \gamma \; \frac{( x^l(n,c) - \bar{x}^l(c)  )}{ \hat{\sigma}^l(c) } }\right \rfloor
\label{eq:standarizedaverages}
\end{equation}

\noindent where $\gamma > 0$ is an input parameter of the algorithm and  it is related to the image scale. In addition, $ x^l(n,c) $ is the point-to-point averaged multichannel EEG signal for the sample point $n$ and for channel $c$. Lastly, $$\bar{x}^l(c) =\frac{1}{n_{max}}\sum_{n=1}^{n_{max}}x^l(n,c)$$ and $$ \hat{\sigma}^l(c) = (\frac{1}{n_{max}-1}\sum_{n=1}^{n_{max}}(x^l(n,c)-\bar{x}^l(c))^2 )^{\frac{1}{2}}$$ are the mean and estimated standard deviation of $x^l(n,c), 1 \leq n \leq n_{max}$, for each channel $c$.

Consequently, a binary image $I^{(l,c)}$ is constructed according to

\begin{equation}
I^{(l,c)}(z_1,z_2) = \left\{ \begin{array}{rl}
255 & \text{if} \,  z_1 = \gamma \  n; \! z_2 = \tilde{x}^l(n,c) + z^l(c) \\
0   & \mbox{otherwise}
\end{array}\right.
\label{eq:images}
\end{equation}

\noindent with $255$ being white and representing the signal's value location and $0$ for black which is the background contrast, conforming a black-and-white plot of the signal.  Pixel arguments $ (z_1,z_2) \in \mathbb{N} \times \mathbb{N}$ iterate over the width (based on the length of the signal segment) and height (based on the peak-to-peak amplitude) of the newly created image with $1 \leq n \leq n_{max}$ and $1 \leq c \leq C$.  The value $z^l(c)$ is the image vertical position where the signal's zero value has to be situated in order to fit the entire signal within the image for each channel c:

\begin{equation}
z^l(c) = \left \lfloor{ \frac{\max_{n} \tilde{x}^l(n,c)  - \min_{n} \tilde{x}^l(n,c) }{2} }\right \rfloor -   \left \lfloor{ \frac{\max_{n} \tilde{x}^l(n,c)  + \min_{n} \tilde{x}^l(n,c)}{ 2} }\right \rfloor
\label{eq:zerolevel}
\end{equation}

\noindent where the minimization and maximization are carried out for $n$ varying between ${1 \leq n\leq n_{max}}$, and $ \lfloor \cdot  \rfloor $ denote the rounding to the smaller nearest integer of the number.
  
In order to complete the plot $I^{(l,c)}$ from the pixels, the Bresenham \cite{Bresenham1965,Ramele2016} algorithm is used to interpolate straight lines between each pair of  consecutive pixels.


\subsubsection{Feature Extraction: Histogram of Gradient Orientations}
\label{SIFT}


For each generated image $I^{(l,c)}$, a keypoint $\mathbf{p_k}$ is placed on a pixel $(x_{p_k}, y_{p_k})$ over the image plot and a window around the keypoint is considered. A local image patch of size $X_p \times X_p$ pixels is constructed by dividing the window in $16$ blocks of size $3s$ each one,  where $s$ is the scale of the local patch and it is an input parameter of the algorithm. It is arranged in a $4 \times 4$ grid and the pixel $ \mathbf{p_k}$ is the patch center, thus $X_p = 12s $ pixels. 

A local representation of the signal shape within the patch can be described by obtaining the gradient orientations on each of the $16$ blocks $B_{i,j}$  with $ 0 \leq i,j \leq 3$ and creating a histogram of gradients.  This technique is based on Lowe's SIFT~\cite{Lowe2004} method, and it is biomimetically inspired in how the visual cortex detects shapes by analyzing orientations~\cite{cogprints561}.   In order to calculate the histogram, the interval $[0,360]$ of possible angles is divided in $8$ bins, each one of $45$ degrees.

 Hence, for each spatial bin $ 0 \leq i,j \leq 3$, corresponding to the indexes of each block $B_{i,j}$,  the orientations are accumulated in a  $3$-dimensional histogram $h$ through the following equation: 
 

\begin{equation}
 h(\theta,i,j) = 3 s \sum_{\mathbf{p} \in I^{(l,c)}} w_\mathrm{ang}(\angle J(\mathbf{p}) - \theta)\, w_{ij}\left(\frac{\mathbf{p} - \mathbf{p_k}}{3 s}\right)\, |J(\mathbf{p})|
\label{eq:histogram}
\end{equation}

\noindent  where $\mathbf{p}$ is a pixel from the image $I^{(l,c)}$,  $\theta$ is the angle bin with $ \theta \in \{0, 45, 90, 135, 180, 225, 270, 315\} $,  $ |J(\mathbf{p})| $ is the norm of the gradient vector in the pixel $\mathbf{p}$ and it is computed using finite differences and $\angle J(\mathbf{p}) $ is the angle of the gradient vector.  The scalar $ w_\mathrm{ang}(\cdot) $  and vector $ w_{ij}(\cdot) $ functions are linear interpolations used by~\cite{Lowe2004} and \cite{Vedaldi2010} to provide a weighting contribution to eight adjacent bins.  They are calculated as  

\begin{equation}
 w_{ij}(\mathbf{v}) = w( v_x - x_i ) w( v_y - y_j ) 
\label{eq:ij}
\end{equation}

\noindent with $ 0 \leq i,j \leq 3$ and

\begin{equation}
 w_\mathrm{ang}(\alpha) = \sum_{r = -1 }^{1} w \bigg( \frac{8\alpha}{2\pi} + 8r \bigg)
\label{eq:wang}
\end{equation}


\noindent where $x_i$ and $y_i$ are the spatial bin centers located in $ x_i,y_j \in \{-\frac{3}{2},-\frac{1}{2},\frac{1}{2},\frac{3}{2}\} $, $\mathbf{v} = ( v_x, v_y ) $ is a vector variable and $\alpha$ a scalar variable.  On the other hand, $r$ is an integer that can vary freely between $ \left[ -1,1 \right]$ which allows the argument $\alpha$ to be unconstrained in terms of its values in radians. The interpolating function $w(\cdot)$ is defined as $ w(z) = \max(0,|z|-1)$.

These binning functions conform a trilinear interpolation that has a combined effect of sharing the contribution of each oriented gradient between their eight adjacent bins in a tridimensional cube in the histogram space, and zero everywhere else.

Lastly, the fixed value of $ 3 $ is a magnification factor which corresponds to the number of pixels per each block when $s = 1$.  As the patch has  $16$ blocks and  $8$ bin angles are considered, for each location $l$ and channel $c$ a feature called \textit{descriptor} $\mathbf{d}^{(l,c)}$  of $128$ dimension is obtained. 
%It can be observed that the histogram is computed by multiplying by $ |J(\mathbf{p})| $, so the method considers both, the magnitude and the orientation of the gradient vector. 

Figure~\ref{fig:sampledescriptor} shows an example of a patch and a scheme of the histogram computation. In (A) a plot of the signal and the patch centered around the keypoint is shown. In (B) the possible orientations on each patch are illustrated.  Only the upper-left four blocks are visible.  The first eight orientations of the first block, are labeled from $1$ to $8$ clockwise. The orientations of the second block $ B_{1,2} $ are labeled from $9$ to $16$.  This labeling continues left-to-right, up-down until the eight orientations for all the sixteen blocks are assigned. They form the corresponding descriptor $\mathbf{d}$ of $128$ coordinates. Finally, in (C) an enlarged image plot is shown where the oriented gradient vector for each pixel can be seen.
 
\subsubsection{Speller Matrix letter Identification}
\label{Classification}

\paragraph{P300 ERP Extraction}
Segments corresponding to row flickering are labeled 1-6, whereas those corresponding to column flickering are labeled 7-12.  The extraction process has the following steps:

\begin{itemize}
%\setcounter{enumi}{3}

\item \textbf{Step A:}\label{pasoa} First highlight rows and columns from the matrix in a random permutation order and obtain the Ensemble Average as detailed in steps~\ref{paso1}, \ref{paso2} and \ref{paso3} in Section \ref{Average}.
\item \textbf{Step B:}\label{paso4} Plot the signals $\tilde{x}^l(n,c)$,  $1 \leq n \leq n_{max}$, $1 \leq c \leq C $,  according Section~\ref{Plot} in order to generate the images $I^{(l,c)}$ for rows and columns $1 \leq l \leq 12$.

\item \textbf{Step C:} Obtain the descriptors $ \mathbf{d}^{(l,c)}$ for rows and columns from $I^{(l,c)}$  in accordance to the method described in Section~\ref{SIFT}. 

\end{itemize}

\paragraph{Calibration}

A trial, as defined by the BCI2000 platform~\cite{Schalk2004}, is every attempt to select just one letter from the speller.  A set of trials is used for calibration and once the calibration is complete it can be used to identify new letters from new trials.

During the calibration phase, two descriptors $ \mathbf{d}^{(l,c)}$ are extracted for each available channel, corresponding to the locations $l$ of a selection of one previously instructed letter from the set of calibration trials.  These descriptors are the P300 templates, grouped together in a template set called $ T^c $.   The set is constructed using the steps described in Section \ref{Average} and the steps A, B and C of the P300 ERP extraction process.

Additionally, the best performing channel, $bpc$ is identified based on the the channel where the best Character Recognition Rate is obtained.

\paragraph{Letter identification}

In order to identify the selected letter, the template set $T^{bpc}$ is used as a database.  Thus, new descriptors are computed and they are compared against the descriptors belonging to the calibration template set $T^{bpc}$.

\begin{itemize}

\item \textbf{Step D:} Match to the calibration template $T^{bpc}$ by computing  

\begin{equation}
\hat{row} = \arg \min_{l \in \{1,\dots,6\}} \sum_{q \in N_T(\mathbf{d}^{(l,bpc)})}^{} {\left\lVert q -  \mathbf{d}^{(l,bpc)} \right\rVert}  ^{2}
\label{eq:multiclassificationrow}
\end{equation}

\noindent and

\begin{equation}
\hat{col} = \arg \min_{l \in \{7,\dots,12\}} \sum_{q \in N_T(\mathbf{d}^{(l,bpc)})}^{} {\left\lVert q -  \mathbf{d}^{(l,bpc)} \right\rVert} ^{2}
\label{eq:multiclassificationcol}
\end{equation}

\noindent where $N_T(\mathbf{d}^{(l,bpc)})$  is defined as $N_T(\mathbf{d}^{(l,bpc)}) = \{\mathbf{d} \in T^{bpc} / $  is the k-nearest neighbor of $ \mathbf{d}^{(l,bpc)} \}$ for the best performing channel.  This set is obtained by sorting all the elements in $T^{bpc}$ based on distances between them and $\mathbf{d}^{(l,bpc)}$, choosing the $k$ with smaller values, with $k$ a parameter of the algorithm.  This procedure is based on the k-NBNN  algorithm~\cite{Boiman2008}.

\end{itemize}
By computing the aforementioned equations, the letter of the matrix can be determined from the intersection of the row $ \hat{row} $ and column $ \hat{col} $. 
Figure~\ref{fig:classification} shows a scheme of this process. 



\subsection{Subsection Heading Here}
Subsection text here.

% needed in second column of first page if using \IEEEpubid
%\IEEEpubidadjcol

\subsubsection{Subsubsection Heading Here}
Subsubsection text here.


% An example of a floating figure using the graphicx package.
% Note that \label must occur AFTER (or within) \caption.
% For figures, \caption should occur after the \includegraphics.
% Note that IEEEtran v1.7 and later has special internal code that
% is designed to preserve the operation of \label within \caption
% even when the captionsoff option is in effect. However, because
% of issues like this, it may be the safest practice to put all your
% \label just after \caption rather than within \caption{}.
%
% Reminder: the "draftcls" or "draftclsnofoot", not "draft", class
% option should be used if it is desired that the figures are to be
% displayed while in draft mode.
%
%\begin{figure}[!t]
%\centering
%\includegraphics[width=2.5in]{myfigure}
% where an .eps filename suffix will be assumed under latex, 
% and a .pdf suffix will be assumed for pdflatex; or what has been declared
% via \DeclareGraphicsExtensions.
%\caption{Simulation results for the network.}
%\label{fig_sim}
%\end{figure}

% Note that the IEEE typically puts floats only at the top, even when this
% results in a large percentage of a column being occupied by floats.


% An example of a double column floating figure using two subfigures.
% (The subfig.sty package must be loaded for this to work.)
% The subfigure \label commands are set within each subfloat command,
% and the \label for the overall figure must come after \caption.
% \hfil is used as a separator to get equal spacing.
% Watch out that the combined width of all the subfigures on a 
% line do not exceed the text width or a line break will occur.
%
%\begin{figure*}[!t]
%\centering
%\subfloat[Case I]{\includegraphics[width=2.5in]{box}%
%\label{fig_first_case}}
%\hfil
%\subfloat[Case II]{\includegraphics[width=2.5in]{box}%
%\label{fig_second_case}}
%\caption{Simulation results for the network.}
%\label{fig_sim}
%\end{figure*}
%
% Note that often IEEE papers with subfigures do not employ subfigure
% captions (using the optional argument to \subfloat[]), but instead will
% reference/describe all of them (a), (b), etc., within the main caption.
% Be aware that for subfig.sty to generate the (a), (b), etc., subfigure
% labels, the optional argument to \subfloat must be present. If a
% subcaption is not desired, just leave its contents blank,
% e.g., \subfloat[].


% An example of a floating table. Note that, for IEEE style tables, the
% \caption command should come BEFORE the table and, given that table
% captions serve much like titles, are usually capitalized except for words
% such as a, an, and, as, at, but, by, for, in, nor, of, on, or, the, to
% and up, which are usually not capitalized unless they are the first or
% last word of the caption. Table text will default to \footnotesize as
% the IEEE normally uses this smaller font for tables.
% The \label must come after \caption as always.
%
%\begin{table}[!t]
%% increase table row spacing, adjust to taste
%\renewcommand{\arraystretch}{1.3}
% if using array.sty, it might be a good idea to tweak the value of
% \extrarowheight as needed to properly center the text within the cells
%\caption{An Example of a Table}
%\label{table_example}
%\centering
%% Some packages, such as MDW tools, offer better commands for making tables
%% than the plain LaTeX2e tabular which is used here.
%\begin{tabular}{|c||c|}
%\hline
%One & Two\\
%\hline
%Three & Four\\
%\hline
%\end{tabular}
%\end{table}


% Note that the IEEE does not put floats in the very first column
% - or typically anywhere on the first page for that matter. Also,
% in-text middle ("here") positioning is typically not used, but it
% is allowed and encouraged for Computer Society conferences (but
% not Computer Society journals). Most IEEE journals/conferences use
% top floats exclusively. 
% Note that, LaTeX2e, unlike IEEE journals/conferences, places
% footnotes above bottom floats. This can be corrected via the
% \fnbelowfloat command of the stfloats package.




\section{Conclusion}
The conclusion goes here.





% if have a single appendix:
%\appendix[Proof of the Zonklar Equations]
% or
%\appendix  % for no appendix heading
% do not use \section anymore after \appendix, only \section*
% is possibly needed

% use appendices with more than one appendix
% then use \section to start each appendix
% you must declare a \section before using any
% \subsection or using \label (\appendices by itself
% starts a section numbered zero.)
%


\appendices
\section{Proof of the First Zonklar Equation}
Appendix one text goes here.

% you can choose not to have a title for an appendix
% if you want by leaving the argument blank
\section{}
Appendix two text goes here.


% use section* for acknowledgment
\section*{Acknowledgment}


The authors would like to thank...


% Can use something like this to put references on a page
% by themselves when using endfloat and the captionsoff option.
\ifCLASSOPTIONcaptionsoff
  \newpage
\fi



% trigger a \newpage just before the given reference
% number - used to balance the columns on the last page
% adjust value as needed - may need to be readjusted if
% the document is modified later
%\IEEEtriggeratref{8}
% The "triggered" command can be changed if desired:
%\IEEEtriggercmd{\enlargethispage{-5in}}

% references section

% can use a bibliography generated by BibTeX as a .bbl file
% BibTeX documentation can be easily obtained at:
% http://mirror.ctan.org/biblio/bibtex/contrib/doc/
% The IEEEtran BibTeX style support page is at:
% http://www.michaelshell.org/tex/ieeetran/bibtex/
\bibliographystyle{IEEEtran}
% argument is your BibTeX string definitions and bibliography database(s)
\bibliography{IEEEabrv,histogram}
%
% <OR> manually copy in the resultant .bbl file
% set second argument of \begin to the number of references
% (used to reserve space for the reference number labels box)
%\begin{thebibliography}{1}

%\bibitem{IEEEhowto:kopka}
%H.~Kopka and P.~W. Daly, \emph{A Guide to \LaTeX}, 3rd~ed.\hskip 1em plus
%  0.5em minus 0.4em\relax Harlow, England: Addison-Wesley, 1999.

%\end{thebibliography}

% biography section
% 
% If you have an EPS/PDF photo (graphicx package needed) extra braces are
% needed around the contents of the optional argument to biography to prevent
% the LaTeX parser from getting confused when it sees the complicated
% \includegraphics command within an optional argument. (You could create
% your own custom macro containing the \includegraphics command to make things
% simpler here.)
%\begin{IEEEbiography}[{\includegraphics[width=1in,height=1.25in,clip,keepaspectratio]{mshell}}]{Michael Shell}
% or if you just want to reserve a space for a photo:

\begin{IEEEbiography}{Michael Shell}
Biography text here.
\end{IEEEbiography}

% if you will not have a photo at all:
\begin{IEEEbiographynophoto}{John Doe}
Biography text here.
\end{IEEEbiographynophoto}

% insert where needed to balance the two columns on the last page with
% biographies
%\newpage

\begin{IEEEbiographynophoto}{Jane Doe}
Biography text here.
\end{IEEEbiographynophoto}

% You can push biographies down or up by placing
% a \vfill before or after them. The appropriate
% use of \vfill depends on what kind of text is
% on the last page and whether or not the columns
% are being equalized.

%\vfill

% Can be used to pull up biographies so that the bottom of the last one
% is flush with the other column.
%\enlargethispage{-5in}



% that's all folks
\end{document}


