Dear Editor, 

We are submitting a Full Paper, entitled "A Control Strategy for a Tethered Follower Robot for Pulmonary Rehabilitation" which we believe is within the scope of IEEE Transactions on Neural Systems and Rehabilitation Engineering .  This article encompass work performed at CiC Laboratory of the ITBA University in Buenos Aires, Argentina, in the context of research on Rehabilitation Robotics.  This is an original production, which has not been submitted for publication elsewhere.

This work is the first approach to implement a solution that was brought to us from ALPI organization.  ALPI is a non-profit civil association located in Buenos Aires, Argentina, that provides neuromotor rehabilitation for pediatric and adult patients. It was founded in 1943 with the main focus of treating children with poliomyelitis, and has since expanded to deal with all kinds of related diseases.  In the context of a translational collaboration agreement with them, they identified the problem stated in this work and we started to work together on a technological and engineering solution.  We describe in these pages a basic scheme that we design to drive a robotic platform that can carry the oxygen tank of emphysema patients that are under Pulmonary Rehabilitation treatments.  Our objective was to implement a gradual iterative design strategy and produce a prototyping platform that tackled the problem of the patient following mechanism while keeping affordable costs and that can receive quick feedback from ALPI care-givers.  We show that the very simple design and controlling scheme is able to implement a following strategy and that the proposed solution is viable.

We believe this is an interesting work that is in response to a practical call received from the medical community.  One of the goals of worldwide rehabilitation engineering is to provide solutions to society health care problems, and we understood that in order to tackle these problems effectively, we need to attain design strategies that allow quick feedback from medical personnel.  We hope this project to be the cornerstone of a fruitful collaboration.

Yours truly
Rodrigo Ramele
ITBA University 
Argentina